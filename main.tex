%% 
%% Copyright 2007-2024 Elsevier Ltd
%% 
%% This file is part of the 'Elsarticle Bundle'.
%% ---------------------------------------------
%%  
%% It may be distributed under the conditions of the LaTeX Project Public
%% License, either version 1.3 of this license or (at your option) any%% later version.  The latest version of this license is in
%%    http://www.latex-project.org/lppl.txt
%% and version 1.3 or later is part of all distributions of LaTeX
%% version 1999/12/01 or later.
%% 
%% The list of all files belonging to the 'Elsarticle Bundle' is
 %% given in the file `manifest.txt'.
%% 
%% Template article for Elsevier's document class `elsarticle'
%% with harvard style bibliographic references

\documentclass[preprint,12pt,authoryear]{elsarticle}

%% Use the option review to obtain double line spacing
%% \documentclass[authoryear,preprint,review,12pt]{elsarticle}

%% Use the options 1p,twocolumn; 3p; 3p,twocolumn; 5p; or 5p,twocolumn
 %% for a journal layout:
%% \documentclass[final,1p,times,authoryear]{elsarticle}
%% \documentclass[final,1p,times,twocolumn,authoryear]{elsarticle}
%% \documentclass[final,3p,times,authoryear]{elsarticle}
%% \documentclass[final,3p,times,twocolumn,authoryear]{elsarticle}
%% \documentclass[final,5p,times,authoryear]{elsarticle}
%% \documentclass[final,5p,times,twocolumn,authoryear]{elsarticle}

%% For including figures, graphicx.sty has been loaded in
%% elsarticle.cls. If you prefer to use the old commands
%% please give \usepackage{epsfig}

%% The amssymb package provides various useful mathematical symbols
\usepackage{amssymb}
\usepackage{graphicx}
%% The amsmath package provides various useful equation environments.
\usepackage{amsmath}
\usepackage{algorithm}
\usepackage{algorithmicx}
\usepackage{algpseudocode}
%% The amsthm package provides extended theorem environments
%% \usepackage{amsthm}

%% The lineno packages adds line numbers. Start line numbering with
%% \begin{linenumbers}, end it with \end{linenumbers}. Or switch it on
%% for the whole article with \linenumbers.
%% \usepackage{lineno}

\journal{Computers and Operation Research}

\begin{document}

\begin{frontmatter}

%% Title, authors and addresses

%% use the tnoteref command within \title for footnotes;
%% use the tnotetext command for theassociated footnote;
%% use the fnref command within \author or \affiliation for footnotes;
%% use the fntext command for theassociated footnote;
%% use the corref command within \author for corresponding author footnotes;
%% use the cortext command for theassociated footnote;
%% use the ead command for the email address,
%% and the form \ead[url] for the home page:
%% \title{Title\tnoteref{label1}}
%% \tnotetext[label1]{}
%% \author{Name\corref{cor1}\fnref{label2}}
%% \ead{email address}
%% \ead[url]{home page}
%% \fntext[label2]{}
%% \cortext[cor1]{}
%% \affiliation{organization={},
%%            addressline={}, 
%%            city={},
%%            postcode={}, 
%%            state={},
%%            country={}}
%% \fntext[label3]{}

\title{Actor-based Large Neighborhood Search for weekly maintenance scheduling} %% Article title

%% use optional labels to link authors explicitly to addresses:
%% \author[label1,label2]{}
%% \affiliation[label1]{organization={},
%%             addressline={},
%%             city={},
%%             postcode={},
%%             state={},
%%             country={}}
%%
%% \affiliation[label2]{organization={},
%%             addressline={},
%%             city={},
%%             postcode={},
%%             state={},
%%             country={}}

\author{Christian Brunbjerg Jespersen} %% Author name
\author{Thomas Jacob Riis Stidsen}
\author{Kristoffer Sigsgaard Wernblad}
\author{Kasper Barslund Hansen}
\author{Jingrui Ge}
\author{Simon Didriksen}
\author{Niels Henrik Mortensen}

%% Author affiliation
\affiliation{organization={Technical University of Denmark},%Department and Organization
            addressline={Anker Egelundsvej 1}, 
            city={Kongens Lyngby},
            postcode={2800}, 
            state={Hovedstaden},
            country={Denmark}}
\affiliation{organization={Technical University of Denmark},%Department and Organization
            addressline={Anker Egelundsvej 1}, 
            city={Kongens Lyngby},
            postcode={2800}, 
            state={Hovedstaden},
            country={Denmark}}
\affiliation{organization={Technical University of Denmark},%Department and Organization
            addressline={Anker Egelundsvej 1}, 
            city={Kongens Lyngby},
            postcode={2800}, 
            state={Hovedstaden},
            country={Denmark}}
\affiliation{organization={Technical University of Denmark},%Department and Organization
            addressline={Anker Egelundsvej 1}, 
            city={Kongens Lyngby},
            postcode={2800}, 
            state={Hovedstaden},
            country={Denmark}}
\affiliation{organization={Technical University of Denmark},%Department and Organization
            addressline={Anker Egelundsvej 1}, 
            city={Kongens Lyngby},
            postcode={2800}, 
            state={Hovedstaden},
            country={Denmark}}
\affiliation{organization={Technical University of Denmark},%Department and Organization
            addressline={Anker Egelundsvej 1}, 
            city={Kongens Lyngby},
            postcode={2800}, 
            state={Hovedstaden},
            country={Denmark}}
\affiliation{organization={Technical University of Denmark},%Department and Organization
            addressline={Anker Egelundsvej 1}, 
            city={Kongens Lyngby},
            postcode={2800}, 
            state={Hovedstaden},
            country={Denmark}}
%% Abstract
\begin{abstract}
%% Text of abstract
Serveral problems facing the operations research field have proven difficult to solve due to their inherrent uncertainty
and highly dynamic nature. Stochastic optimization, fuzzy logic, and robust optimization are some of the methods that 
have been proposed to solve these issues. These methods make an implicit assumption on static data and
a static problem setting. Maintenance scheduling is one such problem where the best available information continually 
updates and then therefore the scheduling continuously needs to be updated. Maintenance scheduling is a complex process and 
it correct implementation is usually more associated with operation management, but this paper will argue that is possible 
to implement general maintenance scheduling approaches if the solution method is designed to be integrated into a business
process of the kind that are usually developed by the principles of operation management. 

This paper proposes a novel optimization method that is capable to optimizing a scheduling problem in the following setting: 
primary data source is changing in real-time; external inputs affects the optimization process; multiple actors
are making interdependent decision whose objectives may differ significantly. The proposed solution approach is an actor-based implementation
of the large neighborhood search metaheuristic and the paper will show that this approach can naturally model the dynamic nature of operational problems.
\end{abstract}

%%Graphical abstract
\begin{graphicalabstract}
%\includegraphics{grabs}
\end{graphicalabstract}

%%Research highlights
\begin{highlights}
\item How to allow direct and real-time integration into an optimization process?
\item How to perform optimization in a real-time changing parameter space?
\end{highlights}

%% Keywords
\begin{keyword}
%% keywords here, in the form: keyword \sep keyword
Large Neighborhood Search \sep Actor Framework \sep Real-time Optimization \sep Human-centered Computing \sep Interactive Systems and Tools \sep Decision Support Systems \sep Interactive Optimization.


%% PACS codes here, in the form: \PACS code \sep code

%% MSC codes here, in the form: \MSC code \sep code
%% or \MSC[2008] code \sep code (2000 is the default)

\end{keyword}

\end{frontmatter}

%% Add \usepackage{lineno} before \begin{document} and uncomment 
%% following line to enable line numbers
%% \linenumbers

%% main text
%%

%% Use \section commands to start a section
\section{Introduction}
\label{sec:1-introduction}
%% Labels are used to cross-reference an item using \ref command.

Maintenance scheduling is a dynamic and operational problem and have proven hard to solve and study in operation research due to the need of tight integration with tacit knowledge of decision makers and 
the way that industry usually assigns responsibility for decision-making 
to an individual representing only a small part of the complete process. 

These multiple smaller processes are often difficult to map to a single mathematical model describing the whole system as elaborated by (\citep{barthelemy_human_2002}).
Solving operation research problems that are operational in nature have additional requirements over conventional static problems: they have to be responsive to changing parameters; 
able to be assimilated into the decision-makers workflow; allow for integration with dynamic data sources such as databases and RESTapi \citep{meignan_review_2015}. 
Operational aspects of operation research, as opposed to higher level strategic and tactical aspects, are characterized by extensive amounts negotiation and feedback on 
proposed schedules. The lack of integration and responsiveness can lead to schedules that are not directly implemented in practice but instead provides intial suggestions \cite{meignan_review_2015}, which are
then iterated else where in the scheduling process. In \citep{barthelemy_human_2002} the authors argue that many problems that operation research aim to solve are often composed of a group of individuals whose decisions are 
consolidated into an "epistemic subject" for which a mathematical model can be formulated and solved, with many scheduling problems being good examples. Furthermore, some multi-objective optimization problems are a product of 
there being multiple actors in the decision making process each with different views on an optimal schedule from their vantage point rather then there being actual multiple objectives for each individual actor.

This paper proposes a solution method that will allow for real-time optimization based on actor/user interaction and connection to a dynamic 
data source, effectively meaning changes to the parameter space. The proposed solution method will be tested on the multi-compartment multi-knapsack problem (MCMKP) on a large dataset from a maintenance performing company. The MCMKP 
naturally models what in the practical maintenance is called the weekly schedule, taken form \citep{palmerMaintenancePlanningScheduling2019}. It should be noted 
that the scientific maintenance scheduling literature deviates significantly from its practical implementation which is detailed in \citep{palmerMaintenancePlanningScheduling2019}. 
The solution method will by based on the large neighborhood search (LNS) metaheuristic. This meta heuristic was chosen due to its properties of naturally being able to work with and fix 
infeasible solutions and its state of the art performance on various scheduling problems. 

The paper is divided into four different sections. Section \ref{sec:2-solution-method} explains the weekly maintenance scheduling model in detail and forms the fundation of the paper. 
Section \ref{sec:3-results} shows that results coming from the implemented system where the implementation will be affected by simulated user-interaction. Section \ref{sec:4-discussion} 
will discuss the implications of the research and possible future research directions.

\subsection{The Weekly Schedule: Multi-compartment Multi-knapsack Problem with capacity penalties}
\label{sub1sec2}
The actor-based large neighborhood search is implemented on the MCMKP which models that weekly schedule in maintenance. 
The model is comprised of 
five different sets. $P$ is the number of weekly periods; $W$ is the number of work orders; $\tau$ is the number of different traits; $E$ is a set that
defines which work orders should be excluded from a specific weekly period; $I$ is an inclusion set that defines the allocation of specific work orders 
which should be included in a specific weekly period. The model has four parameters. $v_{pw}$ is the value of work order $w$ in weekly period $p$; $d$ is the 
penalty for exceeding a specific trait capacity; $c_{w\tau}$ is the capacity requirement for work order $w$ for trait $t$; $cap_{p\tau}$ is the total amount 
of capacity available in for weekly period p for for trait t. The model has 2 decision variables. $x_{wp}$, is a binary decision variable equal to one 
if work order w is in weekly period p and zero otherwise; $pen_{p\tau}$ is non-negative decision variable equal to the amount of excess capacity above 
the $cap_{p\tau}$ in weekly period p for trait $\tau$. The parameters $v$, $cap$, $Q$, and $P$ are functions of time, $\tau$, in this case as they will be 
subject to change during the solution process.

\begin{alignat}{2}
	& \text{Min} \quad \sum_{w = 1}^{W} \sum_{p = 1}^{P} v_{wp}(t) \cdot x_{wp}(t) + \sum_{p = 1}^{P} \sum_{\tau = 1}^T d \cdot pen_{p\tau}(t)   \label{eqn:objective_function_strategic} \\[1em]
    & \text{subject to:} \notag                                                                                                                                        \\[1em]
	& \sum_{w = 1}^W c_{w\tau} \cdot x_{wp}(t) \leq \ cap_{p\tau}(t) + pen_{p\tau}(t)        && \forall p \in P, \forall \tau \in T                      \label{eqn:capacity_constraint}          \\[1em]
	& \sum_{w = 1}^{W} x_{wp}(t) = 1                                            && \forall p \in P                                       \label{eqn:single_workorder_constraint}  \\[1em]
	& x_{wp}(t) = 0                                                             && \forall (w, p) \in E(t)                               \label{eqn:exclusion_constraint}         \\[1em]
	& x_{wp}(t) = 1                                                             && \forall (w, p) \in I(t)                               \label{eqn:inclusion_constraint}         \\[1em]
	& x_{wp}(t) \in \{0, 1\}                                                    && \forall w \in W, \forall p \in P                      \label{eqn:x_integrality_constraint}     \\[1em] 
	& pen_{p\tau}(t) \in \mathbb{R}^{+}                                         && \forall p \in P, \forall \tau \in T                      \label{eqn:p_non_negativity_constraint}
\end{alignat}

The objective function \ref{eqn:objective_function_strategic} minimizes the total weight of all work order assignments together with the penalty $d$ for exceeding the 
capacity given in constraint \ref{eqn:capacity_constraint}. Constraint \ref{eqn:capacity_constraint} ensures that all the weights $c_{w\tau}$ for each activity in an work order, given that it
has been assigned, is lower than the capacity for each period and for each trait $\tau$. $pen_{p\tau}$ is the amount of exceeded capacity that is needed for the current assignment of work order to be feasible.
Constraint \ref{eqn:single_workorder_constraint} makes sure that each work order is assigned to atleast a single period. Constraint \ref{eqn:exclusion_constraint} excludes a work order from a 
certain period and constraint \ref{eqn:inclusion_constraint} forces a specific work order to be in a specific period. Constraint \ref{eqn:x_integrality_constraint} and \ref{eqn:p_non_negativity_constraint} 
specify the variable domain for $x_{wp}$ and $pen_{p\tau}$ respectively. The effects of changing $E$, $I$, $cap$, and $v$ in real-time will be examined to determine their effects on the weekly schedules and objective value.

\section{Solution Method}
\label{sec:2-solution-method}

\subsection{Actor-based Large Neighborhood Search}
A problem which is affected by user-interaction and requires real-time feedback needs an optimization approach that is able to repair infeasible schedules and while also 
converging quickly. For this the large neighborhood search metaheuristic has been shown satisfy these requirements in the literature \cite{gendreau_handbook_2019}. 

The LNS metaheuristic is defined for static problems, meaning that the parameters that make up the problem instance is not subject to change 
after the algorithm has been started. To make the LNS able adapt to changing parameters in real-time a message system have been implemented into the existing framework. This 
extension is shown in algorithm \ref{algo1}.  

\subsubsection{Messages And Destructors}
LNS in its most basic form has one constructor and one destructor which repeatedly destroy and rebuild the schedule. For the AbLNS we will generalize on this concept by 
including messages as destructors of the classic LNS implementation. This generalization can be seen as being somewhat similar to how the adaptive LNS (ALNS) is formulated,
but where the different constructors and destructors are chosen externally as well. 

Extending on the classic setup we define the following set of destructors, $M$:

\begin{itemize}
	\item $m_1$: Inclusion destruct message	
	\item $m_2$: Exclusion destruct message	
	\item $m_3$: Capacities destruct message	
	\item $m_4$: Weights destruct message	
	\item $m_5$: Random destruct message
\end{itemize}

Each of these messages affect different parts of the MCMK problem (weekly schedule). Notice
here that the first four messages destruct the solution by changing the parameter space and the last message is 
a random destructor.

Generalizing the destructors from being static structures into messages
allows the solution to change in real-time to a changing paramenter space meaning
that the algorithm does not need to restart to handle changes in data. 

\begin{algorithm}[H]
\caption{Actor-based Large Neighborhood Search}  \label{algo1}
\begin{algorithmic}[1]
\State \textbf{Input} queue = message queue
\State \textbf{Input} P     = problem instance
\State \textbf{Input} x     = initial schedule
% \State $x^b = x$
\While{true}
	\While{queue.has\_message()}
        % \State $m = queue.pop()$
        % \State $m.destruct(x^b)$
		\State $P.update(m)$
        \State $x.destruct(m)$
    \EndWhile
	
    \State $x^t = x.repair()$
    % \If{accept($x^t$,\ $x$)}                       \label{alg:acceptance_criteria_start}
    %     \State x = x$^t$
    % \EndIf                                         \label{alg:acceptance_criteria_end}
    \If{$c(x^t) < c(x)$}                             \label{alg:objective_start}
        \State $x = x^t$
		\State queue.send($x$)
    \EndIf                                           \label{alg:objective_end}
	\State queue.push($m_5$)
\EndWhile\\
\Return $x^b$
\end{algorithmic}
\end{algorithm}

The basic LNS setup have here been extended with a `message queue`. This message queue will be read from on every iteration of the LNSs main iteration loop. Here we notice that the 
incoming message are able to change both the solutoin but also the problem instance itself. Here we see one of the defining features of the LNS metaheuristic in play, that due to its inherrent 
property of being able to optimize a solution that have become infeasible which is something that is very likely to happen when you change the parameter of the problem instance itself. 

Another less obvious property the message queue allows is for the algorithm to run indefinitely and instead of restarting the algorithm you instead pass 
messages to it to allow it be adjust both the solution space and the parameter space.
This property avoid the issue of time consuming initial convergence as the algorithm will be found in an optimimal state when the solution is perturbed.  

\section{Results}
\label{sec:3-results}
The results section will: 1. introduce the real-world data instance; 2. show the effect of forcing item set in the specific weekly schedules; 3. show the effect of changing the 
period capacities, and 4. show the effect of dynamically changing the value of the work orders $v$. 

\subsection{Data Instance}
\begin{table}[H]
\centering
\begin{tabular}{|c|c|c|c|c|}
\hline
           & \begin{tabular}[c]{@{}c@{}}Number of\\ Item Sets\end{tabular} & \begin{tabular}[c]{@{}c@{}}Number of\\ Compartments\end{tabular} & \begin{tabular}[c]{@{}c@{}}Number of\\ Knapsacks\end{tabular} \\ \hline
Instance 1 & 3487                                                          & 16                                                               & 52                                                            \\ \hline
\end{tabular}

\caption{Table Caption} % \label{fig1}
\end{table}
% \begin{table}[t]%% placement specifier
% %% Use tabular environment to tag the tabular data.
% %% https://en.wikibooks.org/wiki/LaTeX/Tables#The_tabular_environment
% \centering%% For centre alignment of tabular.
% \begin{tabular}{l c r}%% Table column specifiers
% %% Tabular cells are separated by &
%   1 & 2 & 3 \\ %% A tabular row ends with \\
%   4 & 5 & 6 \\
%   7 & 8 & 9 \\
% \end{tabular}
% %% Use \caption command for table caption and label.
% \end{table}

\subsection{Response to Inclusion}
The response to the inclusion of a work order is given by I parameter of the model which 
is constrained in \ref{eqn:inclusion_constraint} of model given in \ref{sub1sec2}.

The inclusion is made of forcing certain allocations of work orders to be in specific periods. Below a table is provided 
to show what changes will occur and at what and at what point in time.
\begin{table}[H]
\centering
\begin{tabular}{|c|c|c|c|c|c|}
\hline
\begin{tabular}[c]{@{}c@{}}\end{tabular}     & \begin{tabular}[c]{@{}c@{}}At Time:\\ 01:00\end{tabular} & \begin{tabular}[c]{@{}c@{}}At Time:\\ 02:00\end{tabular} & \begin{tabular}[c]{@{}c@{}}At Time:\\ 03:00\end{tabular} & \begin{tabular}[c]{@{}c@{}}At Time:\\ 04:00\end{tabular} & \begin{tabular}[c]{@{}c@{}}At Time:\\ 05:00\end{tabular} \\ \hline
\begin{tabular}[c]{@{}c@{}}$\Delta |P|$\end{tabular} & 10                                                       & 20                                                       & 30                                                       & 40                                                       & 50                                                       \\ \hline
\end{tabular}
\end{table}

With the inputs defined we will explain the main results which are shown in the figure below. 
% Use figure environment to create figures
% Refer following link for more details.
% https://en.wikibooks.org/wiki/LaTeX/Floats,_Figures_and_Captions
\begin{figure}[H]%% placement specifier
%% Use \includegraphics command to insert graphic files. Place graphics files in 
%% working directory.
\centering%% For centre alignment of image.
\includegraphics[width=1.0\textwidth]{figures/objective.png}
%% Use \caption command for figure caption and label.
\caption{Figure Caption}\label{fig:response-to-inclusion}
%% https://en.wikibooks.org/wiki/LaTeX/Importing_Graphics#Importing_external_graphics
\end{figure}

\subsection{Response to Exclusion}
\begin{figure}[H]%% placement specifier
%% Use \includegraphics command to insert graphic files. Place graphics files in 
%% working directory.
\centering%% For centre alignment of image.
\includegraphics[width=1.0\textwidth]{figures/objective-400-exclusions.png}
%% Use \caption command for figure caption and label.
\caption{Figure Caption}\label{fig:objective-exclusion-400}
%% https://en.wikibooks.org/wiki/LaTeX/Importing_Graphics#Importing_external_graphics
\end{figure}

\subsection{Response to Changes in Knapsack Capacities}
The effects of changes to capacities will be illustrated in the same way as it was with the response to inclusion and below we see the table that shows which inputs that the AbLNS will be affected by.

\begin{table}[H]
\centering
\begin{tabular}{|c|c|c|c|c|c|}
\hline
                      & \begin{tabular}[c]{@{}c@{}}At Time:\\ 01:00\end{tabular} & \begin{tabular}[c]{@{}c@{}}At Time:\\ 02:00\end{tabular} & \begin{tabular}[c]{@{}c@{}}At Time:\\ 03:00\end{tabular} & \begin{tabular}[c]{@{}c@{}}At Time:\\ 04:00\end{tabular} & \begin{tabular}[c]{@{}c@{}}At Time:\\ 05:00\end{tabular} \\ \hline
$\Delta |p|$ & 16                                                       & 16                                                       & 16                                                       & 16                                                       & 16                                                       \\ \hline
$\Delta |\tau|$ & 16                                                       & 16                                                       & 16                                                       & 16                                                       & 16                                                       \\ \hline
$\Delta |cap|$& 100                                                      & 200                                                      & 400                                                      & 800                                                      & 1600                                                     \\ \hline
\end{tabular}
\end{table}

\begin{figure}[H]%% placement specifier
%% Use \includegraphics command to insert graphic files. Place graphics files in 
%% working directory.
\centering%% For centre alignment of image.
\includegraphics[width=1.0\textwidth]{figures/objective-resource-increases.png}
%% Use \caption command for figure caption and label.
\caption{Figure Caption}\label{fig:objective-resource-increases}
%% https://en.wikibooks.org/wiki/LaTeX/Importing_Graphics#Importing_external_graphics
\end{figure}

Correspondingly we also have the figure below in which the resources are decreasing.

\subsection{Response to Changes in Item Weights}
The final parameter that will be changed is the work order value $v$. This section will be more elaborate as we have to show how that the work orders are rearranged due to the changes in their value across the different periods.

\begin{table}[H]
\centering
\begin{tabular}{|c|c|c|c|c|c|}
\hline
             & \begin{tabular}[c]{@{}c@{}}At Time:\\ 01:00\end{tabular} & \begin{tabular}[c]{@{}c@{}}At Time:\\ 02:00\end{tabular} & \begin{tabular}[c]{@{}c@{}}At Time:\\ 03:00\end{tabular} & \begin{tabular}[c]{@{}c@{}}At Time:\\ 04:00\end{tabular} & \begin{tabular}[c]{@{}c@{}}At Time:\\ 05:00\end{tabular} \\ \hline
$\Delta |w|$ & 20                                                       & 40                                                       & 80                                                       & 160                                                      & 320                                                      \\ \hline
$\Delta |p|$ & 26                                                       & 26                                                       & 26                                                       & 26                                                       & 26                                                       \\ \hline
$\Delta |v|$ & $1 \cdot 10^{5}$                                         & $2 \cdot 10^{5}$                                         & $4 \cdot 10^{5}$                                         & $8 \cdot 10^{5}$                                         & $1.6 \cdot 10^{6}$                                       \\ \hline
\end{tabular}
\end{table}

\section{Discussion}
\label{sec:4-discussion}
\subsection{Integration}
Figure \ref{fig:normal-setup} illustrates
a classic approach to operations research where a model allows the decision maker 
the obtain a solution to from which to make better informed decision. This approach is suitable to static problems and
problems there the parameters that make up the problem change slowly. This approach is often unsatisfactory in environments where
the best available information is ever changing. 

Actor-based large neighborhood search enables tight integration between both the users and a 
dynamic data sources but crucially it naturally allows models to communicate with each other mimicking most 
practical decisions which are made up of multiple actors, as in the maintenance scheduling problem.

\begin{figure}[H]%% placement specifier
%% Use \includegraphics command to insert graphic files. Place graphics files in 
%% working directory.
\centering%% For centre alignment of image.
\includegraphics[width=0.2\textwidth]{figures/normal-setup.pdf}
%% Use \caption command for figure caption and label.
\caption{Traditional use case of developed optimization algorithms. Notice that the optimization algorithm is separated from the data source, requiring user interaction to optimize the process}
\label{fig:normal-setup}
\end{figure}

Figure \ref{fig:actor-setup} illustrates the setup that is enabled by using a actor-based optimization approachs. Due to the implemented message system it becomes possible 
to build large decision making structures from smaller units. 
\begin{figure}[H]%% placement specifier
%% Use \includegraphics command to insert graphic files. Place graphics files in 
%% working directory.
\centering%% For centre alignment of image.
\includegraphics[width=0.65\textwidth]{figures/actor-setup.pdf}
%% Use \caption command for figure caption and label.
\caption{Figure Caption}\label{fig:actor-setup}
\end{figure}

The three most conseqential properties of this approach is: 1. continual optimization saves the initial time it takes to reach converence; 2. due to the message system the 
optimization approach will always be responsive to changes in both the parameter space
and the solution space; 3. due to the encapsulation and message passing properties of the suggested optimization approach it becomes possible to apply the metaheuristic in multi-model 
and hierarchical model setups, providing an
approach for modelling and optimizing large scale systems. 

\subsection{Continuous Optimization}
By continually optimizing it becomes possible to only optimize the pertubations to the schedule. This means that we avoid having to optimize the problem to reach initial convergence.

In many cases this can save a significant amount of computations, refer to \cite{alza_bartlett_ceberio_mccall_2023}, especially if specilized constructors and destructors are implemented to handle the specific pertubations.

\subsection{Message Passing versus Restarts}
There is evidence that the effect of dynamic optimization is not always efficient, \cite{alza_bartlett_ceberio_mccall_2023} mentions the idea that dynamic optimization approaches can be "elusive" in that they do not always provide a speedup over restarting the solution method implementation. 
The paper clearly shows that dynamic approaches are the most effective when changes are small and frequent, which does align with the idea behind the actor-based LNS in that changes should be optimized around when ever they occur. T   


\subsection{System Level Optimization}
This approach could also enable larger scale optimzations providing a modular approach to operations research. This is beneficial 

%% Use \subsubsection, \paragraph, \subparagraph commands to 
%% start 3rd, 4th and 5th level sections.
%% Refer following link for more details.
%% https://en.wikibooks.org/wiki/LaTeX/Document_Structure#Sectioning_commands

%% Displayed equations can be tagged using various environments. 

%% Single line equations can be tagged using the equation environment.
%% Unnumbered equations are tagged using starred versions of the environment.
%% amsmath package needs to be loaded for the starred version of equation environment.
%% align or eqnarray environments can be used for multi line equations.
%% & is used to mark alignment points in equations.
%% \\ is used to end a row in a multiline equation.
%% Unnumbered versions of align and eqnarray

%% Refer following link for more details.
%% https://en.wikibooks.org/wiki/LaTeX/Mathematics
%% https://en.wikibooks.org/wiki/LaTeX/Advanced_Mathematics

%% Use a table environment to create tables.
%% Refer following link for more details.
%% https://en.wikibooks.org/wiki/LaTeX/Tables
% \begin{table}[t]%% placement specifier
% %% Use tabular environment to tag the tabular data.
% %% https://en.wikibooks.org/wiki/LaTeX/Tables#The_tabular_environment
% \centering%% For centre alignment of tabular.
% \begin{tabular}{l c r}%% Table column specifiers
% %% Tabular cells are separated by &
%   1 & 2 & 3 \\ %% A tabular row ends with \\
%   4 & 5 & 6 \\
%   7 & 8 & 9 \\
% \end{tabular}
% %% Use \caption command for table caption and label.
% \caption{Table Caption}\label{fig1}
% \end{table}


%% Use figure environment to create figures
%% Refer following link for more details.
%% https://en.wikibooks.org/wiki/LaTeX/Floats,_Figures_and_Captions
% \begin{figure}[t]%% placement specifier
% %% Use \includegraphics command to insert graphic files. Place graphics files in 
% %% working directory.
% \centering%% For centre alignment of image.
% \includegraphics{example-image-a}
% %% Use \caption command for figure caption and label.
% \caption{Figure Caption}\label{fig1}
% %% https://en.wikibooks.org/wiki/LaTeX/Importing_Graphics#Importing_external_graphics
% \end{figure}


%% The Appendices part is started with the command \appendix;
%% appendix sections are then done as normal sections
% \appendix
% \section{Example Appendix Section}
% \label{app1}

% Appendix text.

%% For citations use: 
%%       \citet{<label>} ==> Lamport (1994)
%%       \citep{<label>} ==> (Lamport, 1994)
%%
% Example citation, See \cite{interactive-optimization-methods} 
%% If you have bib database file and want bibtex to generate the
%% bibitems, please use
%%
%%  \bibliographystyle{elsarticle-harv} 
%%  \bibliography{<your bibdatabase>}

%% else use the following coding to input the bibitems directly in the
%% TeX file.

%% Refer following link for more details about bibliography and citations.
%% https://en.wikibooks.org/wiki/LaTeX/Bibliography_Management
<<<<<<< HEAD
\bibliography{refs}
=======
\bibliography{bib/refs}
>>>>>>> e7eda4191ef254a36c2a136d7e87517183270cac

\bibliographystyle{elsarticle-harv}
\end{document}

\endinput
%%
%% End of file `elsarticle-template-harv.tex'.
